%%%%%%%%%%%%%%%%%
% This is an example CV created using altacv.cls (v1.1, 21 November 2016) written by
% LianTze Lim (liantze@gmail.com), based on the 
% Cv created by BusinessInsider at http://www.businessinsider.my/a-sample-resume-for-marissa-mayer-2016-7/?r=US&IR=T
% 
%% It may be distributed and/or modified under the
%% conditions of the LaTeX Project Public License, either version 1.3
%% of this license or (at your option) any later version.
%% The latest version of this license is in
%%    http://www.latex-project.org/lppl.txt
%% and version 1.3 or later is part of all distributions of LaTeX
%% version 2003/12/01 or later.
%%%%%%%%%%%%%%%%

%% If you want to use \orcid or the
%% academicons icons, add "academicons"
%% to the \documentclass options. 
%% Then compile with XeLaTeX or LuaLaTeX.
% \documentclass[10pt,a4paper,academicons]{altacv}
\documentclass[10pt,a4paper]{altacv}

%% AltaCV uses the fontawesome and academicon fonts
%% and packages. 
%% See texdoc.net/pkg/fontawecome and http://texdoc.net/pkg/academicons for full list of symbols.
%% When using the "academicons" option,
%% Compile with LuaLaTeX for best results. If you
%% want to use XeLaTeX, you may need to install
%% Academicons.ttf in your operating system's font %% folder.


% Change the page layout if you need to
\geometry{left=1cm,right=9cm,marginparwidth=6.8cm,marginparsep=1.2cm,top=1cm,bottom=1cm}

% Change the font if you want to.

% If using pdflatex:
\usepackage[utf8]{inputenc}
\usepackage[T1]{fontenc}
\usepackage[default]{lato}
\usepackage{hyperref}

\hypersetup{
    colorlinks=true,
    linkcolor=cyan,
    filecolor=magenta,      
    urlcolor=blue,
}

% If using xelatex or lualatex:
% \setmainfont{Lato}

% Change the colours if you want to
\definecolor{VividPurple}{HTML}{2F9E43}
\definecolor{SlateGrey}{HTML}{2E2E2E}
\definecolor{LightGrey}{HTML}{666666}
\colorlet{heading}{VividPurple}
\colorlet{accent}{VividPurple}
\colorlet{emphasis}{SlateGrey}
\colorlet{body}{LightGrey}

% Change the bullets for itemize and rating marker
% for \cvskill if you want to
\renewcommand{\itemmarker}{{\small\textbullet}}
\renewcommand{\ratingmarker}{\faCircle}

%% sample.bib contains your publications
\addbibresource{sample.bib}

\begin{document}
\name{MARTIN KELLNER}
  \tagline{  }
\photo{3.5cm}{25}
\personalinfo{%
  % Not all of these are required!
  % You can add your own with \printinfo{symbol}{detail}
  \quad{Born: 24.04.1995}\\
  \location{Bratislava, Slovakia}\\
  \email{martinkellner1@gmail.com}
  \github{https://github.com/martinkellner}
  % \phone{+33666312478}
  
   % I'm just making this up though.
%   \orcid{orcid.org/0000-0000-0000-0000} % Obviously making this up too. If you want to use this field (and also other academicons symbols), add "academicons" option to \documentclass{altacv}
}

%% Make the header extend all the way to the right, if you want. Extend the right margin by 8cm (=6.8cm marginparwidth + 1.2cm marginparsep)
\begin{adjustwidth}{}{-8cm}
\makecvheader

\end{adjustwidth}
\cvsection{EDUCATION}
\cvevent{}{High School - Electrical engineering}
{2014-2010}{High Industrial School of Technology and Design Poprad}
\cvevent{}{Bachelor - Applied Computer Science}
{2014 -- 2017}{Faculty of Mathematics, Physics and Informatics - Comenius University Bratislava}
Bachelor thesis: \href{http://alis.uniba.sk:8088/lib/item?id=chamo:653237&fromLocationLink=false&theme=Katalog}{Collaborative Code Writing in Eclipse} (Slovak only)

\cvevent{} {Master - Applied Computer Science/Artificial Intelligence}
{2017 -- 2019}{Faculty of Mathematics, Physics and Informatics - Comenius University Bratislava}
\github{Master thesis: \href{https://raw.githubusercontent.com/martinkellner/master-thesis/master/thesis/kellner.pdf}{Learning Eye-Hand Coordinate Transformation in a Simulated Humanoid Robot}}

%% Provide the file name containing the sidebar contents as an optional parameter to \cvsection.
%% You can always just use \marginpar{...} if you do
%% not need to align the top of the contents to any
%% \cvsection title in the "main" bar.
\cvsection[page1sidebar]{PROFESSIONAL EXPERIENCES}
\cvevent{Software Developer}{Atos IT Solutions and Services s.r.o.}{September 2017 -- July 2019} {Bratislava}
Student support during my studies. BackEnd and FrontEnd development.
\newline
\begin{itemize}
\item BackEnd Development (Java/Spring, Gradle)
\item FrontEnd Development (Angular2+)
\item CICD (Gradle, Jenkins)
\end{itemize}
\divider
\cvevent{AI/ML Engineer}{Raiffeisen Bank International AG}{August 2019 -- present} {Bratislava}
Development and maintenance of a chatbot platform. PoC's in a few NLP topics.
\newline
\begin{itemize}
\item Sentiment Analysis for the Russian Language - PoC - analyzing potential data sources, text pre-processing for pre-trained models, evaluating results, and presenting results to management.
\item Named entity recognition - PoC - pre-processing of exported emails, labeling data based on regular expressions, and extracting of required information from text using the combination of regular expressions and a NER model built on top of \href{https://github.com/flairNLP/flair}{Flair}.
\item Cooperating with others in developing a chatbot platform - Python3 scripts, Docker (including hardening of images), Helm charts, Bash scripting.
\item Working on a few chatbot projects - Python3 scripts, deploying on Rancher/EKS, Gitlab CICD.
\item Speech2Text for the Slovak language - PoC - Experimenting with the Wav2Vec-U paper from Facebook AI. 
\end{itemize}
\divider
\end{document}
